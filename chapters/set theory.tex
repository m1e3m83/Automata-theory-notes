\chapter{نظریه مجموعه‌ها}

\textbf{
    مجموعه و عضویت
}
:

ما تعاریفی برای مفاهیم پایه‌ای مثل مجموعه و عضویت ارائه نمیدهیم و در تعریف این مفاهیم از شهود خود استفاده میکنیم.

\begin{align*}
    &A \times B = \left\{ (x, y) | x \in A, y \in B\right\}\\
    &R \subseteq A \times B\\
    &f_{A \rightarrow B} \Longleftrightarrow f \subseteq A \times B \wedge \left( \forall x \in A, \forall y, y' \in B :
    \left((x, y) \in f \wedge (x, y') \in f\right) \Rightarrow y = y' \right)
\end{align*}

\textbf{تابع کامل:}
اگر در تابع 
f
هر عضو 
A
به یک عضو 
B
نسبت داده شود به آن تابع کامل میگوییم.


\textbf{تابع پوشا:}
اگر در تابع 
f
هر عضو 
B
به یک عضو 
A
نسبت داده شده باشد به آن تابع پوشا میگوییم.

\textbf{تابع یک‌ به‌ یک:}
اگر 
(رابطه)
وارون تابع 
f
همچنان یک تابع باشد به آن یک تابع یک به یک میگوییم.

اگر تابعی هر سه شرط بالا را داشته باشد به آن یک تناظر یک به یک یا
mapping
گفته میشود.

\textbf{مقایسه اندازه بین دو مجموعه:}
اگر تابع تناظر یک به یک دلخواه بین دو مجموعه 
A
و
B
برقرار باشد میگوییم این دو مجموعه اندازه یا کاردینالیتی یکسان دارند یا
$|A| = |B|$
.

همچنین اگر:
$$
\exists C \subseteq B : |C| = |A| \Longleftrightarrow |B| \geq |A|
$$

و به طور مشابه:
$$
|B| > |A| \Longleftrightarrow |B| \geq |A| \wedge |B| \neq |A|
$$

\textbf{
کاردینالیتی مجموعه توانی:
}
کاردینالیتی مجموعه توانی حاصل از  یک مجموعه اکیدا از آن مجموعه بزرگتر است.
$$
|A| < |2^A|
$$

\begin{proof}
اثبات:
هر تابع کامل دلخواه 
$f:A \rightarrow 2^A$
را درنظر بگیرید. حال
$S \in 2^A$
را به طور زیر میسازیم:

$$
S = \{x | x \notin f(x) \}
$$

در این صورت با برهان خلف میتوان اثبات کرد که مجموعه 
$S$
به هیچ عضوی از مجموعه 
$A$
متناظر نشده و به همین دلیل هیچ تابع 
$f$
یک تناظر یک به یک نیست:

فرض کنید که فرض خلف درست باشد و مجموعه 
$S$
به یک عضو مانند 
$a$
نظیر شده باشد. در این حالت اگر 
$a$
عضو 
$S$
باشد یا نباشد با تعریف 
$S$
به تناقض بر میخوریم.

\end{proof}

\textbf{تعریف اعداد به کمک نظریه مجموعه‌ها:}
در نگاه بسیار رادیکال به نظریه مجموعه ها میتوان خود اعداد را به کمک نظریه مجموعه ها تعریف کرد. در این
روش به کمک مفهوم اندازه مجموعه تعریف شده در بالا
(کاردینالیتی)
اندازه مجموعه تهی را صفر مینامیم.

successor 
هر مجموعه را برابر با مجموعه اعضای آن مجموعه و خود مجموعه تعریف میکنیم. به طور مثال:

\begin{align*}
    &A = \varnothing\\
    \text{successor}(A)= &B = \{\varnothing\}\\
    \text{successor}(B) = &C = \{\{\varnothing\}, \varnothing\}
\end{align*}

حال تمامی اعداد طبیعی
$N_j$
(و منتاظرا اعدد صحیح و غیره)
را میتوان به صورت اندازه های این مجموعه ها تعریف کرد که به مجموعه همه این اندازه‌های تعریف شده 
$\mathbb{N}$
میگوییم.

\textbf{متناهی،‌شمارا،‌ناشمارا:}
اگر اندازه مجموعه برابر با یکی از اعداد طبیعی تعریف شده در بالا باشد به آن شمارش پذیر و متناهی و در
غیر این صورت به آن نامتناهی میگوییم.
اگر اندازه یک مجموعه برابر با خود مجموعه 
$\mathbb{N}$
باشد آنگاه به آن شمارش پذیر و شمارا میگوییم.
اگر بتوان یک مجموعه را استقرایی تعریف کرد این مجموعه شمارا است.

در غیر این صورت به این مجموعه ناشمارا گفته میشود.


\textbf{اندازه مجموعه اعداد طبیعی:}
اندازه مجموعه اعداد طبیعی را برابر مقدار
$\aleph$
تعریف میکنیم. میتوان ثابت کرد که 
$\aleph$
عضو اعداد طبیعی نیست.

دقت کنید که با کمک نظریه مجموعه‌ها میتوان میتوان اعمال پایه مانند جمع را نیز مثلا به صورت اندازه اجتماع دو مجموعه بدون اشتراک تعریف کرد.

\begin{proof}
    اثبات:
    دو مجموعه اعداد زوج 
$E$
    و اعداد فرد
$O$
    را در نظر بگیرید.
    میدانیم که
$|O|=|E|=\aleph$
از طرفی با توجه به تعریف جمع
$\mathbb{N} = O \cap E \longrightarrow \aleph + \aleph =\aleph$

که برای هیچ عدد طبیعی ناصفر درست نیست. پس
$\aleph$
یک عدد طبیعی نیست.
\end{proof}
اثبات:
دو مجموعه اعداد زوج 

و اعداد فرد

را در نظر بگیرید.
میدانیم که


\textbf{قضیه:}
مجموعه اعداد صحیح نامتناهی شمارا است.

\begin{proof}
    اثبات: به تابع زیر دقت کنید.

    \begin{align*}
        &f: \mathbb{Z} \rightarrow \mathbb{N}\\
        &f(z) = 
        \begin{cases}
            2z, z > 0\\
            1 - 2z, z \leq 0\\
        \end{cases}
    \end{align*}
\end{proof}

\textbf{قضیه:}
اگر تمام مجموعه های
$A_k$
مجموعه نامتناهی شمارا باشد،
$$
\bigcup_{k= 1}^\aleph A_i
$$

یک مجموعه نامتناهی شمارا است.


\begin{proof}
    یک نکته بسیار مهم در این مسئله این است که استقرا برای اعداد طبیعی مورد استفاده قرار میگیرد و در این
    مسئله از استقرا استفاده نمیکنیم.    

    اثبات: اثبات قضیه بالا معادل است با اثبات اینکه مجموعه 
    $\mathbb{N}\times \mathbb{N}$
    مجموعه شمارا است.

    \begin{align*}
        &f: \mathbb{N}^2 \rightarrow \mathbb{N}\\
        &f(x, y) = 
        \begin{cases}
            1, (x, y) = (1,1)\\
            f(x, y-1) + y-1, x = 1, y > 1\\
            f(x-1, y+1), x \geq 1, y \geq 1
        \end{cases}
    \end{align*}

    که یک تناظر یک به یک است پس اندازه این مجموعه با اندازه اعداد طبیعی برابر است.
\end{proof}

پس حاصل ضرب هر متنهای مجموعه شمارا 
$(\mathbb{N} \times \dots \times \mathbb{N})$
بنابر استقرا یک مجموعه شمارا است.

\textbf{قضیه:}
مجموعه همه رشته‌های بی‌انتها از صفر و یک
(
    $\{0,1\}^\aleph$
)
ناشمارا است.

\begin{proof}
    اثبات:
    با برهان خلف این مسئله را ثابت میکنیم.

    فرض کنید شماره‌گذاری زیر برای این رشته وجود دارد:

    \begin{align*}
        &r_1 = d_{11}d_{12}d_{13}\dots\\
        &r_2 = d_{21}d_{22}d_{23}\dots\\
        &r_3 = d_{31}d_{32}d_{33}\dots\\
        \vdots
    \end{align*}

    حال اگر تعریف کنیم
    $J_{ij} = \overline{d_{ij}}$
    رشته 
    $r = J_{11}J_{22}J_{33}$
    با هریک از رشته های بالا حداقل در یک بیت متفاوت است بانابراین شمرده نشده و به تناقض با فرض خلف میرسیم.
\end{proof}

نتیجه:
\begin{enumerate}
    \item 
    مجموعه اعداد حقیقی بین صفر و یک ناشمارا است.
    \item
    $|\mathbb{R}| = |(0,1)|$
    (اثبات با دایره مثلثاتی)

    \item
    $|P(\mathbb{N} = |\mathbb{R}|)|$
    (هر زیرمجموعه را به یک رشته از صفر و یک بنابر وجود اعضا نسبت میدهیم)

\end{enumerate}
