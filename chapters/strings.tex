\chapter{نظریه‌ای بر مجموعه رشته‌ای}
اگر
$\Sigma =$
یک مجموعه از الفبا باشد،
یک رشته را میتوان به صورت یک تابع از 
$\mathbb{N}_k$
به
$\Sigma$
بیان کرد که
$k$
در آن طول رشته است.

\begin{align*}
    &\Sigma = {a, b, c}\\
    &\omega = aabbc\\
\end{align*}

رشته به طول صفر در یک زبان را که شامل هیچ حرفی نمیشود را با 
$\epsilon$
یا
$\lambda$
مشخص میکنیم.

\begin{align*}
    &\Sigma^* = \left\{ \omega |\text{باشد}\Sigma \text{زیر رشته متناهی از }\omega \right\}\\
    &\Sigma^\omega = \left\{ \omega |\text{باشد}\Sigma \text{زیر رشته نامتناهی از }\omega \right\}\\
    &\Sigma^\infty = \Sigma^* \cup \Sigma^\omega
\end{align*}

در این درس به طور معمول تنها راجب رشته‌های با طول متناهی صحبت میکنیم.

\textbf{اعمال روی رشته‌ها:}
برای رشته دلخواه 
$\phi$
و
$\rho$
اعمال زیرتعریف میشوند.

اضافه کردن انتهای رشته دوم را به رشته اول به صورت
$\phi.\rho$
نشان میدهیم.
بر اساس این تعریف:

\begin{align*}
    \phi^n = \begin{cases}
        \epsilon \quad ,n = 0\\
        \phi . \phi^{n-1} \quad, n>0
    \end{cases}
\end{align*}

\begin{align*}
    \phi^R = \begin{cases}
        \epsilon \quad, |\phi| = 0\\
        a.\omega \quad, |\phi| > 0, \ \phi = \omega.a
    \end{cases} 
\end{align*}

\textbf{زبان:}
اگر 
$\Sigma^*$
مجموعه همه رشته‌های متنهای با الفبای
$\Sigma$
باشد به هر زیرمجموعه
$L \subseteq \Sigma^*$  
یک زبان از الفبای
$\Sigma$    
میگویند.

\begin{example}
    مثال: برای الفبای 
    $\Sigma = \{a, b, c\}$
    زبان های
    $L_1 = \{a, ab\}$
    و
    $L_2 = \{a^nb^n| n \geq 0\}$
    را میتوان تعریف کرد.
\end{example}

مجموعه‌های 
$\epsilon = \{\epsilon\}$
و
$\varnothing$
نیز،‌ هردو زبان هستند.

\textbf{اپراتور ها روی زبان‌ها:}
\begin{enumerate}
    \item اپراتور های مجموعه‌ها
    \item اپراتور های رشته ها: 
    \begin{align*}
        L_1.L_2 = \{\omega . \phi | \omega \in L_1 \wedge \phi \in L_2\}\\
    \end{align*}

    این اپراتور خاصیت جابه‌جایی ندارد ولی

    \begin{align*}
        &(L_1.L_2).L_3 = L_1.(L_2.L_3) \\
        &L.\epsilon = \epsilon. L = L \qquad \text{(عضو خنثی)}\\
        &L.\varnothing = \varnothing.L = \varnothing\qquad \text{(عضو غالب)}
    \end{align*}

    برای آن برقرار است.

    \begin{align*}
        L^R = \{\omega^R | \omega \in L\}\\
        L^n = \{\omega^n | \omega \in L\}
    \end{align*}

    \item اپراتور ستاره کلینی:
    (مجموعه تمامی رشته‌هایی که اگر آنهارا به چند قطعه تقسیم کنیم هر قطعه عضو زبان باشد)

    \begin{align*}
        L^* = L^0 \cup L^1 \cup \dots\\
        L^+ = L^1 \cup L^2 \cup \dots\\
    \end{align*}

    واضح است که 
    $L^+ \subseteq L^*$
    و حالت تساوی صرفا در زمانی رخ میدهد که زبان
    $L$
    شامل رشته
    $\epsilon$
    باشد.
\end{enumerate}

\textbf{
    گرامر:
}

یک گرامر صوری چهارتایی
$G = <V, \Sigma, S, P>$
که در آن
$V$
مجموعه همه متغیر‌ها است،
$\Sigma$
مجموعه متهاهی ثابت ها
(الفبا)
،
$S$
متغیر شروع و 
$P$
مجموعه قواعد است.

هر قاعده 
$p$
به شکل زوج مرتب
$\alpha \rightarrow \beta$
است که در آن:

\begin{align*}
    &\alpha \in (V \cup \Sigma)^+\\
    &\beta \in (V \cup \Sigma)^*
\end{align*}

\begin{example}
    مثال:

    \begin{align*}
        P:&\\
        &A \rightarrow aABC\\
        &A \rightarrow abc\\
        &bB \rightarrow bb\\
        &bC \rightarrow bc\\
        &cC \rightarrow cc
    \end{align*}
\end{example}

اگر 
$G$
یک گرامر باشد به مجموعه همه رشته‌های تولید شده با آن
$L(G)$
میگویند.